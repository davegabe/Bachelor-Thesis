\chapter{Conclusioni}
	In questo lavoro si sono combinate tecniche di speech processing più tradizionali con le odierne di deep learning al fine di ricercare una forma a spettro ridotto della voce che permetta di ridurre la componente acustica, preservando quella linguistica.
	
	Dalle valutazioni si evince che questo approccio è possibile e che i suoi risultati sono equiparabili a quelli ottenuti usando audio senza riduzioni di spettro. Come sviluppi futuri si ritiene interessante approfondirne l'applicazione in altri campi, come ad esempio nella speech recognition al fine di addestrare modelli anonimizzati, e sviluppare metodi per sfruttare al meglio questa forma facilmente manipolabile per data augmentation.
	