\chapter{Test e valutazioni}
		In questo capitolo verranno descritti i test effettuati, i criteri utilizzati per la valutazione e verranno comparati le tre riduzioni spettrali proposte: SWS, noise vocoded speech e buzz vocoded speech.
		
		\section{Dataset}
		È stato utilizzato il dataset fornito dalla VCC2018 (Voice Conversion Challenge 2018) in quanto riferimento principale per le performance di voice conversion.
		
		Il dataset è formato da 8 source speaker e 4 target speaker, a ciascuno di essi corrispondono 81 audio di frasi lette (in totale circa 5 minuti di durata) con frequenza di campionamento a 22.05 kHz. Al fine di lavorare su dati non paralleli, è necessario effettuare le conversioni proposte dal task "Spoke" della VCC2018.
		
		Sono stati usati un sottoinsieme di speaker in modo da testare conversioni tra generi differenti e tra lo stesso genere, elencate a seguire:
		\begin{itemize}
			\item VCC2SF3 $\rightarrow$ VCC2TF1 (F$\rightarrow$F)
			\item VCC2SF3 $\rightarrow$ VCC2TM1 (F$\rightarrow$M)
			\item VCC2SM3 $\rightarrow$ VCC2TF1 (M$\rightarrow$F)
			\item VCC2SM3 $\rightarrow$ VCC2TM1 (M$\rightarrow$M)
		\end{itemize}
		I nomi corrispondono a quelli assegnati internamente del dataset e descrivono il genere (M, F) e l'appartenenza all'insieme di source (S) o target (T).
		
		\section{Training}
		Il dataset è stato processato come descritto nel \autoref{chap:architettura-proposta}. In particolare sono stati predisposti quattro test separati per valutare le seguenti rappresentazioni: nessuna riduzione spettrale, vocoder con noise carrier, vocoder con buzz carrier (500 Hz) e sine-wave speech con 5 formanti.
		
		Le reti sono state trainate per 150k iterazioni seguendo le specifiche della MaskCycleGAN-VC ovvero usando un ottimizzatore Adam, con learning rate impostato a 0.0002 per il generatore e 0.0001 per il discriminatore e momentum $\beta1$ e $\beta2$ rispettivamente 0.5 e 0.999. La batch size è stata impostata a 1, dove ogni sample consiste in 64 frame tagliati casualmente. La maschera applicata sull'input è stata impostata a 25.
		
		\section{Metodi di valutazione}
		Esistono varie metriche per la valutazione oggettiva della voice conversion, in questo lavoro si valuteranno come per la MaskCycleGAN-VC le seguenti: mel-cepstral distortion (MCD)\cite{mel-cepstral-distance} e Kernel DeepSpeech Distance (KDSD)\cite{kdsd}.
		
		La MCD misura la variazione di spettro, di conseguenza non è necessariamente correlata con la naturalezza del suono, mentre la KDSD misura la distanza tra le feature degli audio reali e quelli generati e ha dimostrato una correlazione con la valutazione effettuata da persone.
		Data la natura di queste misurazioni, una conversione è considerabile migliore quando queste metriche hanno risultato più basso.
		
		Verrà inoltre impiegata una MOSNet\cite{mosnet} pre-trainata per ottenere una stima di riferimento per quanto riguarda la valutazione soggettiva. Esso è un modello che, basandosi sulle opinioni reali fornite da ascoltatori alle submission effettuate alla VCC2018, è stato addestrato a predire valutazioni soggettive MOS (mean opinion score) e ha dimostrato una forte correlazione con i voti effettivi delle persone. La valutazione è espressa, come per la MOS, in una valutazione nell'intervallo [1,5] dove 1 significa una bassa qualità e 5 un'ottima qualità.
		
		\section{Risultati}
		Si riportano a seguire i risultati ottenuti dalle conversioni di voci.
		Vengono riportate le valutazioni basate sulle scale MCD, KDSD e MOSNet delle quattro conversioni effettuate per ciascuna tipologia di dati usata come input della rete.
		
		Per ogni tipologia di conversione (es. F$\rightarrow$M) sono stati messi in evidenza i risultati migliori per ciascuna metrica. Risulta interessante notare come il metodo originale \cite{MaskCyclegan-VC} senza riduzioni di spettro, ottenga ottime risultati per la metrica di KDSD, mentre le conversioni effettuate con le  riduzioni di spettro ottengono punteggi migliori per le metriche di MCD e MOSNet.
		
		\def\arraystretch{1.3}
\begin{table}[h]
	\centering
	\begin{threeparttable}
	\begin{tabular*} {\textwidth}{c @{\extracolsep{\fill}} llccc}
		\toprule
		\textbf{Riduzione spettro}\tnote{a}         & \textbf{Valutazione} & \textbf{F$\rightarrow$F} & \textbf{F$\rightarrow$M} & \textbf{M$\rightarrow$F} & \textbf{M$\rightarrow$M} \\ \hline
		\multirow{3}{*}{Nessuna riduzione \tnote{b}}        & MCD (dB) & 6.61          & 6.57          & 6.98          & 6.89          \\
		& KDSD     & \textbf{2074} & \textbf{1755} & \textbf{2770} & \textbf{1583} \\
		& MOSNet   & 3.84          & 4.46          & 3.92          & 4.58          \\ \hline
		\multirow{3}{*}{Noise vocoded\tnote{c}}     & MCD [dB] & 6.53          & \textbf{6.47} & 6.75          & 6.73          \\
		& KDSD [$\times 10^5$]     & 3269          & 2247          & 3446          & 2032          \\
		& MOSNet   & 3.90          & 4.46          & 3.89          & 4.49          \\ \hline
		\multirow{3}{*}{Buzz vocoded\tnote{c}}  & MCD [dB]             & \textbf{6.49}              & 6.49                       & \textbf{6.70}              & \textbf{6.71}              \\
		& KDSD [$\times 10^5$]     & 3063          & 2155          & 3169          & 1823          \\
		& MOSNet   & 3.80          & 4.47          & \textbf{3.94} & 4.53          \\ \hline
		\multirow{3}{*}{Sine-wave speech\tnote{c}}  & MCD [dB] & 6.55          & 6.55          & 6.98          & 6.78          \\
		& KDSD [$\times 10^5$]     & 3513          & 2621          & 4802          & 2492          \\
		& MOSNet   & \textbf{3.91} & \textbf{4.48} & 3.86          & \textbf{4.61} \\
		\bottomrule
	\end{tabular*}
	\begin{tablenotes}
		\item[a] Modulo di riduzione dello spettro applicato sui dati di input.
		\item[b] Nessuna riduzione spettrale applicata, modello trainato come proposta da Kaneko et al. in \cite{MaskCyclegan-VC}.
		\item[c] Metodi di riduzione spettrale proposti, come descritti nella \autoref{sec:proposta-riduzione}.
	\end{tablenotes}
\end{threeparttable}
\end{table}
